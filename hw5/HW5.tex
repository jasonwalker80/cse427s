\documentclass{article}
\usepackage[utf8]{inputenc}
\usepackage[margin=0.5in]{geometry}
\usepackage{amsmath}
\usepackage{amsfonts}
\usepackage{amssymb}
\usepackage{graphicx}
\usepackage{hyperref}
\hypersetup{
    colorlinks=true,
    linkcolor=blue,
    filecolor=magenta,      
    urlcolor=cyan,
}
 
\urlstyle{same}

\begin{document}
\begin{flushleft}
Names: Bob Skowron, Jason Walker\\
Keys: rskowron, jwalker\\
SVN: jwalker: \url{https://svn.seas.wustl.edu/repositories/jwalker/cse427s_fl17/}\\
\end{flushleft}

\begin{itemize}
\item[1.] 
	\begin{itemize}
	\item[a.] The data format is a flat text file where each record represents an HTTP protocol request served by the server generating the log file taking requests from hosts whose IP is recorded within the log record. The information stored in each record is:\\
		\begin{itemize}
		\item IP Address
		\item Timestamp (date and time)
		\item URL accessed
		\item HTTP response code
		\item Elapsed time of the request
		\end{itemize}
	%hadoop fs -cat testlog/test_access_log | head -5
	The test log has 5000 rows and the full data set has 4,477,843, so our sample is approximately .11\% of the total.	
	
	\item[b.] Mapper Input:\\
		(byteoffset, 10.223.157.186 - - [15/Jul/2009:21:24:17 -0700] "GET /assets/img/media.jpg HTTP/1.1" 200 110997)\\
		(byteoffset, 10.223.157.186 - - [15/Jul/2009:21:24:18 -0700] "GET /assets/img/pdf-icon.gif HTTP/1.1" 200 228)\\
		(byteoffset, 10.216.113.172 - - [16/Jul/2009:02:51:28 -0700] "GET / HTTP/1.1" 200 7616)\\
		(byteoffset, 10.216.113.172 - - [16/Jul/2009:02:51:29 -0700] "GET /assets/js/lowpro.js HTTP/1 .1" 200 10469)\\
		(byteoffset, 10.216.113.172 - - [16/Jul/2009:02:51:29 -0700] "GET /assets/css/reset.css HTTP/1.1" 200 1014)\\
		
	Mapper Output:\\
		(10.223.157.186, 1)\\
		(10.223.157.186, 1)\\
		(10.216.113.172, 1)\\
		(10.216.113.172, 1)\\
		(10.216.113.172, 1)\\

	Reducer Input:\\
		(10.223.157.186, [1, 1])\\
		(10.216.113.172, [ 1, 1, 1])\\
		
	Reducer Output:\\
		(10.223.157.186, 2)\\
		(10.216.113.172, 3)\\

	The reducer is summing over an array of ones, same as we saw with the word count examples.
	
	\item[c.] N/A
	\item[d.]
		\begin{itemize}
		\item[i.] Running the job locally, your inputs and outputs are local files and folders, respectively. When running on the cluster, the inputs and outputs are stored in hdfs. Print outputs, when run locally, are written out to the console for the end user to review. When run on the cluster, those outputs are not written out to the user. Running locally, there is no job management. The current job is what is run. When run on the cluster, hadoop (namely YARN) has to manage the resources on the cluster.
		\item[ii.]
		\textit{hadoop jar logfileanalysis.jar stubs.ProcessLogs -fs=file:/// -jt=local ~/workspace/log\_file\_analysis/src/test\_log\_file output\_test}
		\item[iii.] Using Eclipse is a personal preference. The first reason is the JAR file does not need to be compiled and stored on the filesystem. Eclipse is smart enough to use the classes within the project currently under development. This reduces the chances of forgetting to recompile the JAR after editing. The second reason is that command line arguments and inputs can easily be passed to the Run Configuration without having to tell hadoop to use the local filesystem. Each configuration can then be edited or customized as needed without having to go to/from the command line.
		\end{itemize}
	\item[e.]
	%hadoop jar logfileanalysis.jar stubs.ProcessLogs testlog/test_access_log testlog/output_test
	%hadoop fs -cat testlog/output_test/part-r-00000 | wc -l
	Based on the output, there are 10 unique IP addresses in the testlog file. Yes, every line contributed to a count. The sum of the counts in the output file sum to 5000, which is the original number of line items
	\item[f.]
	%hadoop jar logfileanalysis.jar stubs.ProcessLogs weblog/access_log weblog/output_ipcount
	When running the job on an actual cluster we need to keep in mind:
		\begin{itemize}
		\item Size of the input file we will be processing as well as the number of records.
		\item Number of reducers we are using to process the expected number of records.
		\item Communication cost for both Mapper and Reducer based on the numer of records.
		\item Compute resources, ex. slots, CPUs, RAM, etc. available to the cluster.
		\item Storage of both intermediate as well as final MR output on HDFS.
		\end{itemize}
	%to get results
	%hadoop fs -cat weblog/output_ipcount/part-r-00000 | wc -l
	%hadoop fs -cat weblog/output_ipcount/part-r-00000 | grep -e 10.1.100.199 -w -e 10.1.100.5 -w -e 10.99.99.58 -w
	\begin{itemize}
	\item Total number of IP Addresses: 333,923
	\item 10.1.100.199: 35
	\item 10.1.100.5: 1
	\item 10.99.99.58: 21
	\end{itemize}
	The results are globally sorted because the text sort done in the shuffle and sort step...	blah blah blah
	
	\end{itemize}
	
\pagebreak

\item[2.]

	
\end{itemize}

\end{document}
