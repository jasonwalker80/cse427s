\documentclass{article}
\usepackage[utf8]{inputenc}
\usepackage[margin=0.5in,includehead,includefoot]{geometry}
\usepackage{amsmath}
\usepackage{amsfonts}
\usepackage{amssymb}
\usepackage{graphicx}
\usepackage{hyperref}
\usepackage{fancyhdr}
\hypersetup{
    colorlinks=true,
    linkcolor=blue,
    filecolor=magenta,      
    urlcolor=cyan,
}
\begin{document}
\pagestyle{fancy}
\renewcommand{\labelitemiii}{$\bullet$}
\renewcommand{\headrulewidth}{0pt}  no line in header area
\lhead{Names: Bob Skowron, Jason Walker\\
Keys: rskowron, jwalker\\
SVN: jwalker: \url{https://svn.seas.wustl.edu/repositories/jwalker/cse427s_fl17/}} Right header
\urlstyle{same}


\begin{itemize}

\item[1.] 
	\begin{itemize}
		\item[a.]
		%hadoop fs -cat /louacre/weblogs/* | wc -l
		\item[b.] Total lines: 1,079,891; Number of lines with HTML requests: 474,360\\
		Code: \\
		\textbf{Total Lines}: mydata.map(lambda line: line.split(' ')).keyBy(lambda fields: (fields{[0]} + "/" + fields{[2]})).count()\\
		\textbf{Lines with HTML}: mydata.map(lambda line: line.split(' ')).filter(lambda fields: "html" in fields[6]).keyBy(lambda fields: (fields[0] + "/" + fields[2])).count()
	\end{itemize}

\pagebreak
\setlength{\headsep}{5pt}
\item[2.]
	\begin{itemize}
		\item[a.] The keys are the paths to the files. The value is the entire contents of the file.\\
				mydata.keys().take(2)\\
				{[u'hdfs://localhost:8020/loudacre/activations/2008-10.xml',\\
 				u'hdfs://localhost:8020/loudacre/activations/2008-11.xml']}
		\item[b.] flatMap()\\
				import xml.etree.ElementTree as ElementTree\\
				\\
				def getactivations(s):\\
    				filetree = ElementTree.fromstring(s)\\
    				return filetree.getiterator('activation')\\
				\\
				xmldata = mydata.flatMap(lambda fields: getactivations(fields[1]))\\
		\item[c.]
				def getmodel(activation):\\
 					return activation.find('model').text\\
				def getaccount(activation):\\
      				return activation.find('account-number').text\\
      				\\
				xmldata.map(lambda activation: getaccount(activation) + ":" + getmodel(activation)).saveAsTextFile("/loudacre/account-models")

	\end{itemize}
	
\pagebreak
\vspace{2cm}
\item[3.]
	% slide 52/56 on the intro to spark pdf
	\begin{itemize}
		% also collapsing multiple RDDs into a single stage
		\item[a.] Pipelining = Feature of Spark that when possible, Spark will perform sequences of transformations by row so no data is stored. There are several benefits. First, no intermediate records or RDDs have to be stored. Second, it will only process the requisite transformations for the data required. We do not have to apply all transformations to all rows if they end up filtered below e.g. 
		\item[b.] You can pipeline a map and a filter or pipeline and take or count???
	\end{itemize}

\end{itemize}

\end{document}
