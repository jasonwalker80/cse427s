\documentclass{article}
\usepackage[utf8]{inputenc}
\usepackage[margin=0.5in,includehead,includefoot]{geometry}
\usepackage{amsmath}
\usepackage{amsfonts}
\usepackage{amssymb}
\usepackage{graphicx}
\usepackage{hyperref}
\usepackage{fancyhdr}
\hypersetup{
    colorlinks=true,
    linkcolor=blue,
    filecolor=magenta,      
    urlcolor=cyan,
}

\pagestyle{fancy}
\renewcommand{\labelitemiii}{$\bullet$}
\renewcommand{\headrulewidth}{0pt}  %no line in header area
\lhead{Names: Bob Skowron, Jason Walker\\
Keys: rskowron, jwalker\\
SVN: jwalker: \url{https://svn.seas.wustl.edu/repositories/jwalker/cse427s_fl17/}} %Right header
\urlstyle{same}

\begin{document}
\begin{itemize}

\item[1.] 
	\begin{itemize}
		\item[a.] mydata.map(lambda line: line.split(' ')).filter(lambda fields: "html" in fields[6]).keyBy(lambda fields: (fields[0] + "/" + fields[2])).keys().saveAsTextFile("loudacre\textunderscore weblogs\textunderscore html")\\
			\\
			hadoop fs -cat loudacre\textunderscore weblogs\textunderscore html/part-00000 \text{\textbar} head -10\\
			\\
			3.94.78.5/69827\\
			19.38.140.62/21475\\
			129.133.56.105/2489\\
			217.150.149.167/4712\\
			209.151.12.34/45922\\
			184.97.84.245/144\\
			233.60.251.2/33908\\
			160.134.139.204/51340\\
			19.209.18.222/13392\\
			230.220.223.28/12643\\

		%hadoop fs -cat /louacre/weblogs/* | wc -l
		\item[b.] Total lines: 1,079,891; Number of lines with HTML requests: 474,360\\
		Code: \\
		\textbf{Total Lines}: mydata.map(lambda line: line.split(' ')).keyBy(lambda fields: (fields[0] + "/" + fields[2])).count()\\
		\textbf{Lines with HTML}: mydata.map(lambda line: line.split(' ')).filter(lambda fields: "html" in fields[6]).keyBy(lambda fields: (fields[0] + "/" + fields[2])).count()
	\end{itemize}

\pagebreak
\setlength{\headsep}{5pt}
\item[2.]
	\begin{itemize}
		\item[a.] The keys are the paths to the files. The value is the entire contents of the file.\\
				mydata.keys().take(2)\\
				{[u'hdfs://localhost:8020/loudacre/activations/2008-10.xml',\\
 				u'hdfs://localhost:8020/loudacre/activations/2008-11.xml']}
		\item[b.] flatMap()\\
				import xml.etree.ElementTree as ElementTree\\
				\\
				def getactivations(s):\\
    				filetree = ElementTree.fromstring(s)\\
    				return filetree.getiterator('activation')\\
				\\
				xmldata = mydata.flatMap(lambda fields: getactivations(fields[1]))\\
		\item[c.]
				def getmodel(activation):\\
 					return activation.find('model').text\\
				def getaccount(activation):\\
      				return activation.find('account-number').text\\
      				\\
				xmldata.map(lambda activation: getaccount(activation) + ":" + getmodel(activation)).saveAsTextFile("/loudacre/account-models")

	\end{itemize}
	
\pagebreak
\vspace{2cm}
\item[3.]
	% slide 52/56 on the intro to spark pdf
	\begin{itemize}
		% also collapsing multiple RDDs into a single stage
		\item[a.] Spark works in stages. Pipelining is a feature of Spark that when possible, Spark will optimize transformations not separated by shuffles by executing them in a single task. This allows it to skip adding another stage and it can execute those steps on a single cluster node. There are several benefits. First, the intermediate records are not saved to memory or written to disk since the entire process can be executed on a single cluster node. Secondly, because the transformations only have to be executed on a single cluster node, there is no overhead from moving the data around during a shuffle phase. This can improve performance as we have seen that the shuffle phase is often quite expensive.
		\item[b.] Maps and filters can be pipelined together. Any transformation that does not require a shuffle of the data should be able to be pipelined.
	\end{itemize}

\end{itemize}

\end{document}
