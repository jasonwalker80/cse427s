\documentclass{article}
\usepackage[utf8]{inputenc}
\usepackage[margin=0.5in,includehead,includefoot]{geometry}
\usepackage{amsmath}
\usepackage{amsfonts}
\usepackage{amssymb}
\usepackage{graphicx}
\usepackage{hyperref}
\usepackage{fancyhdr}
\hypersetup{
    colorlinks=true,
    linkcolor=blue,
    filecolor=magenta,      
    urlcolor=cyan,
}

\pagestyle{fancy}
\renewcommand{\headrulewidth}{0pt} % no line in header area
\lhead{Names: Bob Skowron, Jason Walker\\
Keys: rskowron, jwalker\\
SVN: jwalker: \url{https://svn.seas.wustl.edu/repositories/jwalker/cse427s_fl17/}\\}% Right header

\urlstyle{same}
\renewcommand{\labelitemiii}{$\bullet$}

\begin{document}
\begin{itemize}
\item[1.] 
	\begin{itemize}
		\item[a.] The \textit{InputFormat} to use is KeyValueTextInputFormat. The default split is tab so no extra settings necessary.\\
			Code to add to driver: \textit{gob.setInputFormatClass(KeyValueTextInputFormat.class}\\
			To get the file path, we use the context to get the current input split and then read the path and name\\
			\textit{FileSplit fileSplit = (FileSplit)context.getInputSplit();}\\
			\textit{String filename = fileSplit.getPath().getName();}\\
		\item[b.] See SVN
		\item[c.] Mapper output for lines from Hamlet:\\
			have	hamlet@282\\
			heaven	hamlet@282\\
			and	hamlet@282\\
			earth	hamlet@282\\
			there	hamlet@133\\
			are	hamlet@133\\
			more	hamlet@133\\
			things	hamlet@133\\
			in	hamlet@133\\
			heaven	hamlet@133\\
			and	hamlet@133\\
			earth	hamlet@133\\
	\end{itemize}

\pagebreak
\item[2.]
	\begin{itemize}
		\item[a.] Cosine similarity: $\frac{r_{x}^{T}\cdot r_{y}}{\parallel r_{x} \parallel \cdot \parallel r_{y} \parallel}$\\
		 This is equivalent to: 
		 	$\frac{\sum_{s \epsilon S_{xy}} r_{xs}*r_{ys}}{\sqrt{\sum_{s \epsilon S_{xy}} r_{xs}^{2}} * \sqrt{\sum_{s \epsilon S_{xy}} r_{ys}^{2}}}$\\
		 Normalizing (i.e. subtracting the relevant means) gives:
		 	$\frac{\sum_{s \epsilon S_{xy}} (r_{xs} - \bar{r_{x}})*(r_{ys}-\bar{r_{y}})}{\sqrt{\sum_{s \epsilon S_{xy}} (r_{xs} - \bar{r_{x}})^{2}} * \sqrt{\sum_{s \epsilon S_{xy}} (r_{ys}-\bar{r_{y}})^{2}}}$\\
		 	Which is the Pearson correlation.
		\item[b.] 
			\begin{itemize}
				\item An advantage of normalization is to remove bias. We can remove users who are overly critical (all low scores) or overly enthusiastic (all high rankings)
				\item With Pearson correlation you need to compute and store the average rating for each user. This requires us to process the entire set of ratings.
			\end{itemize}
		\item[c.] One disadvantage of the Jaccard similarity is that it does not take into account the value of the rating. Thus, even if two users rate the exact same items with opposite ratings it would have a high Jaccard similarity ranking. To overcome this problem, you could initially group by rating and then perform the similarity measure. 
	\end{itemize}

\pagebreak
\item[3.]
	\begin{itemize}
		% slide 25
		\item[a.] The dual approach can be more efficient if the \#users $\gg$ \#items. Also, it can be easier to find similar items.
		\item[b.]
			\begin{tabular}{|c||c|c|c|c|}
			\hline 
				\textbf{User} & \textbf{Movie1} & \textbf{Movie2} & \textbf{Movie3} & \textbf{Movie5}\\
			\hline
				\textbf{user1} & 1 & 3 & 2 & \\
				\textbf{user2} &  & 2 & 3 & 5\\
				\textbf{user3} & 1 & 2 &  & \\
			\hline
			\end{tabular}

			Mapper Output (pairs of movies and ratings):\\
			((movie1,movie2), (1,3))\\
			((movie1,movie3), (1,2))\\
			((movie2,movie3), (3,2))\\
			((movie2,movie3), (2,3))\\
			((movie3,movie5), (3,5))\\
			((movie1,movie2), (1,2))\\
			
			Reducer Input:\\
			((movie1,movie2), [(1,3),(1,2)])\\ 
			((movie1,movie3), (1,2))\\
			((movie2,movie3), [(3,2),(2,3)])\\
			((movie3,movie5), (3,5))\\
			((movie1,movie2), (1,2))\\
			
			Reducer Output (similarity measure of movie pairs):\\
			((movie1,movie2), .99)\\
			((movie1,movie3), 0)\\
			((movie1,movie5), 0)\\
			((movie2,movie3), .92)\\
			((movie2,movie5), 0)\\
			((movie3,movie5), 0)\\
			
	\end{itemize}
	
\end{itemize}

\end{document}
